\section*{Homework2}
\subsection*{Problem 1}
\begin{equation*}
p(terrorist)=0.95,p(discovert)=0.95,p(discoverc)=0.95,p(actualt)=0.1
\end{equation*}
Then:
\begin{equation*}
p(isTerrorist)=\frac{p(actualt)\cdot p(discovert)}{p(actualt)\cdot p(discovert)+p(\overline{actualt})\cdot p(\overline{discoverc}}=\frac{0.0095}{0.0095+0.0495}=0.1610
\end{equation*}

\subsection*{Problem 2}
\begin{align*}
p(rr)=\frac{1}{2},p(three-red-drawn)=1^{3}+(\frac{1}{2})^{3}\\
p(rr\mid three-red-drawn) = \frac{p(rr\cap three-red-drawn}{p(three-red-drawn)} = \dfrac{\frac{1}{4}}{\frac{1}{4}+\frac{1}{16}} = \frac{4}{5} = 0.8
\end{align*}

\subsection*{Problem 3}
$p(u)=\frac{1}{11}$\\
$p(black)=\frac{u}{10}$\\
$p(white)=p(\overline{black})=1-\frac{u}{10}$\\
$p(n_{B}=3 = \sum_{i=0}^{10}(\frac{i}{10})^{3}(1-\frac{i}{10})^{7}\binom{10}{3}\frac{1}{11}$

Then:
\begin{align*}
p(u\mid n_{B}=3)&=\frac{p(u\cap n_{B}=3)}{p(n_{B}=3}\\
				&=\dfrac{(\frac{ui}{10})^{3}(1-\frac{u}{10})^{7}\binom{10}{3}\frac{1}{11}}{\sum_{i=0}^{10}(\frac{i}{10})^{3}(1-\frac{i}{10})^{7}\binom{10}{3}\frac{1}{11}}\\
				&=\dfrac{(\frac{ui}{10})^{3}(1-\frac{u}{10})^{7}}{\sum_{i=0}^{10}(\frac{i}{10})^{3}(1-\frac{i}{10})^{7}}
\end{align*}

\subsection*{Problem 5}\
\[p(x) =
  \begin{cases}
    1	& \forall x\in [0,1]		\\
    0	& else
  \end{cases}
\]\\
\textbf{Calculation of mean}
\begin{equation*}
E(x)=\mu=\int_{0}^{1}xdx=\left[\frac{1}{2}x^2\right]_{0}^{1}=\frac{1}{2}
\end{equation*}

\textbf{Calculation of variance}
\begin{align*}
Var(x)	&=\int_{0}^{1}+(x-\frac{1}{2})^2 dx\\
		&=\int_{0}^{1}x^{2}-x+\frac{1}{4}dx\\
		&=\left[\frac{1}{3}x^{3}-\frac{1}{2}x^{2}+\frac{1}{4}x\right]_{0}^{1}\\
		&=\frac{1}{3}-\frac{1}{2}+\frac{1}{4}=\frac{1}{12}
		&=\frac{1}{12}
\end{align*}

\subsection*{Problem 6}
\subsubsection*{Part a}
Show the following:
\begin{equation}
E(X)=E_{Y}(E_{X\mid Y}(X)
\end{equation}

\textbf{Proof}\linebreak
LS:
\begin{align*}
E(X)=\int_{-\infty}^{\infty}x\cdot f(x)dx
\end{align*}

RS: Let $E_{X\mid Y}(X)$ be denoted as $g(x)$:
\begin{align*}
E_{Y}(g(x))	&=\int_{-\infty}^{\infty}g(X)\cdot f(y)dy	\\
			&=\int_{-\infty}^{\infty}E_{X\mid Y}(X)\cdot f(y)dy	\\
			&=\int_{-\infty}^{\infty}\left(\int_{-\infty}^{\infty}x\cdot f_{x}(x\mid y)dx\right)\cdot f(y)dy	\\
			&=\int_{-\infty}^{\infty}\int_{-\infty}^{\infty}x\frac{f(x,y)}{f(y)\cdot f_{y}(y)}dxdy\\
			&=\int_{-\infty}^{\infty}\left(\int_{-\infty}^{\infty}x\cdot f(x,y)dy\right)dx\\
			&=\int_{-\infty}^{\infty}x\cdot f(x)dx\qquad\square
\end{align*}

\subsubsection*{Part b}
Show the following:
\begin{equation*}
Var(X)=E_{Y}(Var_{X\mid Y}(X))+Var_{Y}(E_{X\mid Y}(X))
\end{equation*}

\textbf{Proof:}

\subsection*{Problem 7}
Proof the following:
\begin{equation*}
P(x\geq a)\leq \frac{E(x)}{a}
\end{equation*}

\textbf{Proof}
\begin{align*}
E(X)	&=\int_{-\infty}^{\infty}x\cdot P(x)dx	\\
		&=\int_{-\infty}^{a}x\cdot P(x)dx+\int_{a}^{\infty}x\cdot P(x)dx	\\
		&\geq\int_{a}^{\infty}x\cdot P(x)dx	\qquad\mid a\leq x\\
		&\geq\int_{a}^{\infty}a\cdot P(x)dx	\\
		&=a\int_{a}^{\infty}P(x)dx	\\
		&=a\cdot P(x\geq a)\quad\square
\end{align*}

For $a\frac{3}{4}$ the inequality resolves to:
\begin{equation*}
P(x\geq \frac{3}{4})\leq \dfrac{\frac{1}{2}}{\frac{3}{4}}=\frac{2}{3}=0.6667
\end{equation*}

\subsection*{Problem 8}
Proof the following:
\begin{equation*}
P(\mid X-E(X)\mid\geq a)\leq\frac{Var(X)}{a^{2}}
\end{equation*}

\textbf{Proof:}
\begin{align*}
(X-E(X))^{2}> 0\Rightarrow P((X-E(X))^{2}\geq a^{2}) \leq \frac{E(X-E(X))^{2}}{a^{2}}=\frac{Var(X)}{a^{2}}
\end{align*}

From:
\begin{equation*}
(X-E(X))^{2}\geq a^{2}\Longleftrightarrow \mid X-E(X)\mid\geq a\Rightarrow P(\mid X-E(X)\mid\geq a)\leq\frac{Var(X)}{a^{2}}\quad\square
\end{equation*}

Given $a=\frac{3}{4}$:
\begin{equation*}
P(\mid X-E(X)\mid\geq \frac{3}{4})\leq\dfrac{(\frac{1}{2}-\frac{1}{2})^{2}}{\left(\frac{3}{4}\right)^{2}} =0
\end{equation*}

\subsubsection*{Problem 9}
Proof the following by induction:
\begin{align*}
f(\lambda_{i}x_{1}+_{\cdots}+\lambda_{n}x_{n}\leq \lambda_{1}\cdot f(x_{1})+_{\cdots}+\lambda_{n}\cdot f(x_{n})
\end{align*}

Following is known:
\begin{align}
\text{f convex}\Rightarrow f(a\cdot x+(1-a)\cdot y)\leq a\cdot f(x)+(1-a)f(y)\\
\lambda_{1},_{\cdots},\lambda_{n}< 0\\
\sum_{i=1}^{n}\lambda_{i}=1\\
\sum_{i=1}^{n}\frac{\lambda_{i}}{1-\lambda_{1}}=1
\end{align}
Because of (1) proposition is true for $n=2$

Hypothesis:
\begin{equation}
p(\lambda_{1}x_{1}+\lambda_{2}x_{2})\leq \lambda_{1}p(x_{1})+\lambda_{2}p(x_{2})\qquad\forall x_{1},x_{2}
\end{equation}

Then:
\begin{align*}
p(\sum_{i=1}^{n+1}\lambda_{i}\cdot x_{i}) = p(\lambda_{1}x_{1}+(1-\lambda_{1})\sum_{i=2}^{n+1}\frac{\lambda_{i}}{1-\lambda_{1}}x_{i} &\leq\lambda_{1}p(x_{1})+(1-\lambda_{1})\sum_{i=2}^{n+1}\frac{\lambda_{i}}{1-\lambda_{1}}p(x_{i}) \qquad\mid (4)\\
p(\lambda_{1}x_{1}+(1-\lambda_{1})\sum_{i=2}^{n+1}x_{i}&\leq \lambda_{1}p(x_{1})+(1-\lambda_{1})\sum_{i=2}^{n+1}p(x_{i})\qquad\mid\sum_{i=2}^{n}\lambda_{i}=1-\lambda_{i}	\\
p(\lambda_{1}x_{1}+\left(\sum_{i=2}^{n}\lambda_{i}\right)\sum_{i=2}^{n+1}x_{i}&\leq \lambda_{1}p(x_{1})+\left(\sum_{i=2}^{n}\lambda_{i}\right)\sum_{i=2}^{n+1}p(x_{i})	\\
p(\lambda_{1} x_{1}+\sum_{i=2}^{n+1}\lambda_{i} x_{i} &\leq \lambda_{1} p(x_{1})+\sum_{i=2}^{n+1}\lambda_{i} p(x_{i}) \quad\square
\end{align*}