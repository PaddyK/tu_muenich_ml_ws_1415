\section*{Homework 1 - Linear Algebra}
\subsection*{Problem 1}
Calculate the eigenvectors and eigenvalues for the matrix:
\begin{equation*}
A=\begin{pmatrix}
2	&	-1	&	0	\\
-1	&	2	&	-1	\\
0	&	-1	&	2
\end{pmatrix}
\end{equation*}

\subsubsection*{Calculation of eigenvalues}
\begin{align*}
det(A-\lambda I) 	&= \begin{vmatrix}
2-\lambda	&	-1	&	0	\\
-1	&	2-\lambda	&	-1	\\
0	&	-1	&	2-\lambda
\end{vmatrix}\\
					&=(2-\lambda)(2-\lambda)(2-\lambda)+(-1)(-1)\cdot 0+0\cdot(-1)(-1)-0\cdot(2-\lambda)\cdot 0 -(2-\lambda)\\
					&\ \cdot(-1)(-1)-(-1)(-1)(2-\lambda)\\
					&=(2-\lambda)^{3}-(2-\lambda)-(2-\lambda)\\
					&=(2-\lambda)^{3}-2(2-\lambda)\\
					&=(2-\lambda)((2-\lambda)^{2}-2)\\
					&=(2-\lambda)(2-4\lambda+\lambda^{2})= 0\Longrightarrow\lambda_{1}=2
\end{align*}
\begin{align*}				
\lambda_{2,3}	&= \frac{4\pm\sqrt{16-8}}{4}\\
				&= 2\pm\sqrt{2}					
\end{align*}

\subsubsection*{Calculation of eigenvector to eigenvalue $\lambda^{1}=2$}
\begin{align*}
(A-2\cdot I)\overrightarrow{x}=\begin{pmatrix}
0	&	-1	&	0\\
-1	&	0	&	-1\\
0	&	-1	&	0
\end{pmatrix} \begin{pmatrix}
x_{1}\\
x_{2}\\
x_{3}
\end{pmatrix} = 0\Rightarrow x_{2}=0,x_{1}=-x_{3}\Rightarrow\overrightarrow{x}=\begin{pmatrix}
1\\
0\\
-1
\end{pmatrix}
\end{align*}

\subsubsection*{Calculation of eigenvector to eigenvalue $lambda_{2}=2+\sqrt{2}$}
\begin{align*}
(A-2+\sqrt{2}\cdot I)\overrightarrow{y}	&=\begin{pmatrix}
									-\sqrt{2}	&	-1	&	0\\
									-1	&	-\sqrt{2}	&	-1\\
									0	&	-1	&	-\sqrt{2}
								  \end{pmatrix} \begin{pmatrix}
									y_{1}\\
									y_{2}\\
									y_{3}
							 	  \end{pmatrix}\\
								&=\begin{pmatrix}
								  -\sqrt{2}	&	-1	&	0\\
									-1	&	-\sqrt{2}	&	-1\\
									0	&	0	&	2
								  \end{pmatrix} \begin{pmatrix}
									y_{1}\\
									y_{2}\\
									y_{3}
								  \end{pmatrix}\\
								&=\begin{pmatrix}
								  -\sqrt{2}	&	-1	&	0\\
									0	&	1	&	\sqrt{2}\\
									0	&	0	&	2
								  \end{pmatrix} \begin{pmatrix}
								    y_{1}\\
									y_{2}\\
									y_{3}
								  \end{pmatrix}\\
								&\Rightarrow x_{3}=2,x_{2}=-2\sqrt{2},x_{1}=2\Rightarrow\overrightarrow{y}=\begin{pmatrix}
								  2\\-2\sqrt{2}\\2
								  \end{pmatrix}
\end{align*}

\subsubsection*{Calculation of eigenvector to eigenvalue $lambda_{2}=2-\sqrt{2}$}
\begin{align*}
(A-2-\sqrt{2}\cdot I)\overrightarrow{z}	&=\begin{pmatrix}
									\sqrt{2}	&	-1	&	0\\
									-1	&	\sqrt{2}	&	-1\\
									0	&	-1	&	\sqrt{2}
								  \end{pmatrix} \begin{pmatrix}
									z_{1}\\
									z_{2}\\
									z_{3}
							 	  \end{pmatrix}\\
								&=\begin{pmatrix}
								  \sqrt{2}	&	-1	&	0\\
									-1	&	\sqrt{2}	&	-1\\
									0	&	0	&	2
								  \end{pmatrix} \begin{pmatrix}
									z_{1}\\
									z_{2}\\
									z_{3}
								  \end{pmatrix}\\
								&=\begin{pmatrix}
								  \sqrt{2}	&	-1	&	0\\
									0	&	2	&	-\sqrt{2}\\
									0	&	0	&	2
								  \end{pmatrix} \begin{pmatrix}
									z_{1}\\
									z_{2}\\
									z_{3}
								  \end{pmatrix}\\
								&\Rightarrow x_{3}=2,x_{2}=2\sqrt{2},x_{1}=2\Rightarrow\overrightarrow{y}=\begin{pmatrix}
								  2\\2\sqrt{2}\\2
								  \end{pmatrix}
\end{align*}

\subsection*{Problem 2}
Proof that following holds:
\begin{equation*}
B=UDU^{-1}
\end{equation*}
$\lambda_{1},_{\cdots},\lambda_{n}\text{ eigenvalues of B},\overrightarrow{x_{1}},_{\cdots},\overrightarrow{x_{n}}\text{ linearly independent eigenvectors}$

\begin{equation*}
U=\begin{pmatrix}x_{1}	&	\cdots	&x_{n}\end{pmatrix},B=\begin{pmatrix}
	\lambda_{1}	&			&	0 \\
				&	\ddots	&	 \\
	0			&			&	\lambda_{n}
	\end{pmatrix}
\end{equation*}

Let $e_{1},_{\cdots},e_{n}$ denote the unit vectors:
\begin{equation*}
\overrightarrow{e_{1}}=\begin{pmatrix}1\\0\\ \vdots\\0\end{pmatrix},_{\cdots},
\overrightarrow{e_{n}}=\begin{pmatrix}0\\ \vdots\\0\\1\end{pmatrix}
\end{equation*}

Because $\overrightarrow{x_{1}},_{\cdots},\overrightarrow{x_{n}}$ are linearly independent they can be expressed as follows:
\begin{equation*}
x_{i}=\alpha_{1}\overrightarrow{e_{1}}+_{\cdots}+\alpha_{1}\overrightarrow{e_{n}}\qquad\alpha\in\texttt{R}
\end{equation*}

Additionally following is true:
\begin{equation}\label{eq:eq3}
U\cdot e_{1} = x_{1},_{\cdots},U\cdot e_{n}=x_{n}\Rightarrow U\cdot e_{i}=x_{i}\Rightarrow e_{i}=U^{-1}x_{1},\quad i=1,_{\cdots},n
\end{equation}

\subsubsection{Proof}
\begin{equation*}
B=UDU^{-1}\Leftrightarrow B\cdot x=UDU^{-1}\cdot x
\end{equation*}
LS:
\begin{equation}\label{eq:p2ls}
Bx=\alpha_{1}Bx_{1}+_{\cdots}+\alpha_{n}Bx_{n}=\alpha_{1}\lambda_{1}x_{1}+_{\cdots}+\alpha_{n}\lambda_{n}x_{n}
\end{equation}

RS:\begin{equation}\label{eq:p2rs}
RS:	UDU^{-1}x=UD(\alpha_{1}U^{-1}x_{1}+_{\cdots}+\alpha_{n}U^{-1}x_{n})=UD(\alpha_{1}e_{1}+_{\cdots}+\alpha_{1}e_{1})
\end{equation}

Since $D$ symmetrical following holds:
\begin{align*}
De_{i}	&=\lambda_{i}e_{i},\quad i=1,_{\cdots},n\quad\mid \text{From \ref{eq:p2ls} and \ref{eq:p2rs}}\\
		&=U(\alpha_{1}\lambda_{1} e_{1}+_{\cdots}+\alpha_{n}\lambda_{n} e_{n})\quad\mid \text{From \ref{eq:eq3}}\\
		&=\alpha_{1}\lambda_{1} x_{1}+_{\cdots}+\alpha_{n}\lambda_{n} x_{n}\\
	LS	&= RS\quad\square
\end{align*}

\subsection*{Problem 4}
\subsection*{Part a}
Proof that following holds:
\begin{equation*}
det(B)=\prod_i=1_{n}\lambda_{i}
\end{equation*}

\textbf{Proof}
\begin{align}
det(UDU^{-1})	&=det(U)\cdot det(B)\cdot \frac{1}{det(U^{-1})}\\
				&=det(D)\\
				&=\lambda_{1}\cdot,_{\cdots}\cdot \lambda_{n}\quad\square
\end{align}

\subsubsection*{Part b}
Proof that following holds:
\begin{equation*}
trace(B)=\sum_{i=1}^{n}\lambda_{i}
\end{equation*}

\textbf{Proof}:
\begin{align*}
tr(A\cdot B)&=\sum_{i=1}^{n}a_{1i}\cdot b_{i1}+_{\cdots}+\sum_{j=1}^{n}a_{nj}\cdot b_{jn}\\
			&=\sum_{i=1}^{n}\sum_{j=1}^{n}a_{ji}\cdot b_{ij}\\
			&=\sum_{i=1}^{n}\sum_{j=1}^{n} b_{ij}\cdot a_{ji}\\
			&=\sum_{i=1}^{n}b_{i1}\cdot a_{1i}+_{\cdots}+\sum_{j=1}^{n}b_{jn}\cdot a_{nj}
\end{align*}
Then following is valid:
\begin{align*}
trace(B)	&=\sum_{i=1}^{n}B_{ii}\\
			&=\sum_{i=1}^{n}(UDU^{-1)}_{ii}\\
			&=\sum_{i=1}^{n}D_{ii}\\
			&=\lambda_{1}+_{\cdots}+\lambda_{n}\quad\square
\end{align*}

\subsection*{Problem 5}
\subsubsection*{Part b}
Show that following holds:
\begin{equation*}
\omega^{T}x^{0}=\omega^{0}
\end{equation*}

\textbf{Proof:}
\begin{equation*}
h(x)\equiv\omega^{0}\omega^{T}x\Rightarrow \omega^{T}x^{0}=\omega^{0}
\end{equation*}
\subsubsection*{Part b}
Show that following holds:
\begin{equation*}
\omega^{T}(x^{1}-x_{2})=0
\end{equation*}
\textbf{Proof:}
\begin{align*}
\omega_{0}+\omega^{T}x_{1}	&=0=\omega_{0}+\omega^{T}x_{2}\\
\omega_{0}+\omega^{T}x_{1}-(\omega_{0}+\omega^{T}x_{2})&=0\qquad\mid :\omega_{0}\\
\omega^{T}x_{1}-\omega^{T}x_{2}&=0\\
\omega^{T}(x^{1}-x_{2})&=0\quad\square
\end{align*}